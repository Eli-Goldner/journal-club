\documentclass[usenames,dvipsnames,pdf]{beamer}

\usepackage{textcomp}
\usepackage{pifont}
\usepackage[utf8]{inputenc}
\usepackage{amsfonts}
\usepackage{amstext}
\usepackage{amsmath}
\usepackage{fancyhdr}
\usepackage{amsthm}
\usepackage{epsfig}
\usepackage{graphicx}
\usepackage{multicol}
\usepackage{cite}
\usepackage{natbib}
% \usepackage{tikz}
\usepackage{bussproofs}
\usepackage{stmaryrd}
\usepackage[tableaux]{prooftrees}
\usepackage{qtree}
\usepackage{mathtools}
\usepackage{scalerel,stackengine}
\usepackage[all]{xy}
% \usetikzlibrary{automata, positioning, shapes, arrows}
% \usepackage[dvipsnames]{xcolor}

\usetheme{CambridgeUS}

%\useoutertheme{miniframes} % Alternatively: miniframes, infolines, split
%\useinnertheme{circles}

%\definecolor{UBCblue}{rgb}{0.04706, 0.13725, 0.26667} % UBC Blue (primary)

% \usecolortheme[named=UBCblue]{structure}
% \usecolortheme[named=RoyalBlue]{structure}
\usecolortheme{seahorse}

% \usecolortheme{beaver}
%\setbeamercolor{spruce}{fg=cyan!90!black}

%\setbeamertemplate{itemize item}{\color{teal}$\blacktriangleright$}
%\setbeamertemplate{itemize subitem}{\color{teal}$\blacktriangleright$}

\renewcommand{\phi}{\varphi}

% \newcommand{\newState}[4]{\node[state,#3](#1)[#4]{#2};}
% \newcommand{\newTransition}[4]{\path[->] (#1) edge [#4] node {#3} (#2);} 
\renewcommand*\linenumberstyle[1]{(#1)}
\def\apeqA{\SavedStyle\sim}
\def\apeq{\setstackgap{L}{\dimexpr.5pt+1.5\LMpt}\ensurestackMath{%
  \ThisStyle{\mathrel{\Centerstack{{\apeqA} {\apeqA}}}}}}

\def\dis{\displaystyle}

\def\QQ{\mathbb Q}
\def\ZZ{\mathbb Z}
\def\RR{\mathbb R}
\def\CC{\mathbb C}
\def\FF{\mathbb F}
\def\NN{\mathbb N}
\def\AA{\mathbb A}
\def\II{\mathbb I}

\def\Cc{\mathcal C}
\def\Dd{\mathcal D}
\def\Pp{\mathcal P}

\def\Af{\mathfrak A}
\def\Bf{\mathfrak B}
\def\Cf{\mathfrak C}
\def\Df{\mathfrak D}
\def\Ef{\mathfrak E}
\def\Ff{\mathfrak F}
\def\Gf{\mathfrak G}
\def\Hf{\mathfrak H}
  
% define 2x2 matrix:
\newcommand\twodmatrix[4]{ \ensuremath{ \left( 
	\begin{array}{cc}
		#1 & #2  \\
		#3 & #4 
	\end{array}  
	\right) } }
  

%%%%%%%%%%%%%%%%%%%%%%%%%%%%%%%%%%%%%%%%%%%%%
\setbeamerfont{footnote}{size=\tiny}

\DeclareMathSymbol{:}{\mathord}{operators}{"3A}

\mode<presentation>{}
%% preamble
\title{A sequence-to-sequence approach for document-level relation extraction}
\author{John Giorgi, Gary D. Bader, Bo Wang}
\begin{document}
	%% title frame
	\begin{frame}
		\titlepage
	\end{frame}


        \section{Overview}

        \begin{frame}{Introduction}
          \begin{itemize}
          \item
            Novel end-to-end joint learning approach for inter-sentence relation extraction.\footnote{Document-level is a stretch, due to encoder limit of 512 tokens they did paragraphs.}
          \item
            Utilizes sequence to sequence architecture.
          \item
            Representation schema for coreferent entities, $n$-ary relations, and disjoint spans in output.
          \end{itemize}
        \end{frame}

        \begin{frame}{Introduction}
          \begin{itemize}
          \item
            New benchmarks for end-to-end results over some biomedical datasets.
          \item
            Competitive results against more complex architectures for datasets with established end-to-end results.
          \end{itemize}
        \end{frame}

        \begin{frame}{Defining Terms}
          End-to-end RE:
          \begin{itemize}
          \item
            Relation extraction depends on entities.
          \item
            Pipeline methods (current standard), use one or more models for NER, and one or more models for RE over discovered entities.
          \item
            End-to-end approaches use one model (possibly with a classification head) to discover the relations, relying on internal representations for entity information.
          \end{itemize}
        \end{frame}

        \begin{frame}{Defining Terms}
          Coreference:
          \begin{itemize}
          \item
            The same entity may have one or more mentions in a given text unit (type vs. token).
          \item
            Test
          \end{itemize}
        \end{frame}
        
        \begin{frame}{Motivation}
          \begin{itemize}
          \item
            Lots of entity and relation information at the document and cross document level.
          \item
            Generalizing sentential pipeline methods (the current standard) for inter-sentential RE is very tricky.\footnote{e.g. our NER/RE system for radiotherapy.}
          \item
            Lots of information takes the form of $n$-ary relations and disjoint spans
          \end{itemize}
        \end{frame}
        
        \section{Method}
        
        \section{Results}
        
        \section{Analysis}
        
        \section{Conclusion}

        \begin{frame}{Conclusion}
        \end{frame}

        \begin{frame}[allowframebreaks]
          \frametitle{References}
          \bibliographystyle{acl}
          %\nocite{quantified}
          %\nocite{hansen2007tableau} 
          %\nocite{bolander2009terminating}
          %\nocite{bolander2007termination}
          %\nocite{hungar1995if}
          %\nocite{bhat1998tableau}
          \bibliography{references}
        \end{frame}
      \end{document}
      
      
